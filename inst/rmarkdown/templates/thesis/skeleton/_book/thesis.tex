% This is a LaTeX thesis template for wage University.
% to be used with Rmarkdown
% This template was produced by Rob Hyndman and modified by M A Siraji
% Version: 1 Nobember 2021

\documentclass{wagethesis}

%%%%%%%%%%%%%%%%%%%%%%%%%%%%%%%%%%%%%%%%%%%%%%%%%%%%%%%%%%%%%%%
% Add any LaTeX packages and other preamble here if required
%%%%%%%%%%%%%%%%%%%%%%%%%%%%%%%%%%%%%%%%%%%%%%%%%%%%%%%%%%%%%%%

\author{Your name}
\title{Thesis Title}
\degrees{}
\def\degreetitle{Doctor of Philosophy}
% Add subject and keywords below
\hypersetup{
     %pdfsubject={The Subject},
     %pdfkeywords={Some Keywords},
     pdfauthor={Your name},
     pdftitle={Thesis Title},
     pdfproducer={Bookdown with LaTeX}
}


\bibliography{thesisrefs}

\begin{document}

\pagenumbering{roman}

\titlepage

%{\setstretch{1.2}\sf\tighttoc\doublespacing}
%\listoftables
%\listoffigures
\hypertarget{preface}{%
\chapter*{Preface}\label{preface}}
\addcontentsline{toc}{chapter}{Preface}

The material in Chapter \ref{ch:intro} has been submitted to the journal \emph{Journal of Impossible Results} for possible publication.

The contribution in Chapter \ref{ch:chapter2} of this thesis was presented in the International Symposium on Nonsense held in Dublin, Ireland, in July 2015.

\newpage
\setstretch{1.2}\sf\tighttoc\doublespacing
\newpage
\listoftables
\newpage
\listoffigures

\clearpage\pagenumbering{arabic}\setcounter{page}{0}

\hypertarget{ch:intro}{%
\chapter{General introduction}\label{ch:intro}}

This is where you introduce the main ideas of your thesis, and an overview of the context and background.

In a PhD, Chapter 2 would normally contain a literature review. Typically, Chapters 3--5 would contain your own contributions. Think of each of these as potential papers to be submitted to journals. Finally, Chapter 6 provides some concluding remarks, discussion, ideas for future research, and so on. Appendixes can contain additional material that don't fit into any chapters, but that you want to put on record. For example, additional tables, output, etc.

\hypertarget{ch:chapter2}{%
\chapter{Title}\label{ch:chapter2}}

This chapter contains a summary of the context in which your research is set.

Imagine you are writing for your fellow PhD students. Topics that are well-known to them do not have to be included here. But things that they may not know about should be included.

Resist the temptation to discuss everything you've read in the last few years. And you are not writing a textbook either. This chapter is meant to provide the background necessary to understand the material in subsequent chapters. Stick to that.

You will need to organize the literature review around themes, and within each theme provide a story explaining the development of ideas to date. In each theme, you should get to the point where your ideas will fit in. But leave your ideas to later chapters. This way it is clear what has been done beforehand, and what new contributions you are making to the research field.

All citations should be done using markdown notation as shown below. This way, your bibliography will be compiled automatically and correctly.

\hypertarget{ch:Chapter3}{%
\chapter{Chapter3}\label{ch:Chapter3}}

This chapter contains a summary of the context in which your research is set.

Imagine you are writing for your fellow PhD students. Topics that are well-known to them do not have to be included here. But things that they may not know about should be included.

Resist the temptation to discuss everything you've read in the last few years. And you are not writing a textbook either. This chapter is meant to provide the background necessary to understand the material in subsequent chapters. Stick to that.

You will need to organize the literature review around themes, and within each theme provide a story explaining the development of ideas to date. In each theme, you should get to the point where your ideas will fit in. But leave your ideas to later chapters. This way it is clear what has been done beforehand, and what new contributions you are making to the research field.

All citations should be done using markdown notation as shown below. This way, your bibliography will be compiled automatically and correctly.

\hypertarget{ch:chapter4}{%
\chapter{Chapter 4}\label{ch:chapter4}}

This chapter contains a summary of the context in which your research is set.

Imagine you are writing for your fellow PhD students. Topics that are well-known to them do not have to be included here. But things that they may not know about should be included.

Resist the temptation to discuss everything you've read in the last few years. And you are not writing a textbook either. This chapter is meant to provide the background necessary to understand the material in subsequent chapters. Stick to that.

You will need to organize the literature review around themes, and within each theme provide a story explaining the development of ideas to date. In each theme, you should get to the point where your ideas will fit in. But leave your ideas to later chapters. This way it is clear what has been done beforehand, and what new contributions you are making to the research field.

All citations should be done using markdown notation as shown below. This way, your bibliography will be compiled automatically and correctly.

\hypertarget{ch:chapter5}{%
\chapter{Chapter 5}\label{ch:chapter5}}

This chapter contains a summary of the context in which your research is set.

Imagine you are writing for your fellow PhD students. Topics that are well-known to them do not have to be included here. But things that they may not know about should be included.

Resist the temptation to discuss everything you've read in the last few years. And you are not writing a textbook either. This chapter is meant to provide the background necessary to understand the material in subsequent chapters. Stick to that.

You will need to organize the literature review around themes, and within each theme provide a story explaining the development of ideas to date. In each theme, you should get to the point where your ideas will fit in. But leave your ideas to later chapters. This way it is clear what has been done beforehand, and what new contributions you are making to the research field.

All citations should be done using markdown notation as shown below. This way, your bibliography will be compiled automatically and correctly.

\hypertarget{ch:chapter6}{%
\chapter{Chapter 6}\label{ch:chapter6}}

This chapter contains a summary of the context in which your research is set.

Imagine you are writing for your fellow PhD students. Topics that are well-known to them do not have to be included here. But things that they may not know about should be included.

Resist the temptation to discuss everything you've read in the last few years. And you are not writing a textbook either. This chapter is meant to provide the background necessary to understand the material in subsequent chapters. Stick to that.

You will need to organize the literature review around themes, and within each theme provide a story explaining the development of ideas to date. In each theme, you should get to the point where your ideas will fit in. But leave your ideas to later chapters. This way it is clear what has been done beforehand, and what new contributions you are making to the research field.

All citations should be done using markdown notation as shown below. This way, your bibliography will be compiled automatically and correctly.

\hypertarget{ch:chapter7}{%
\chapter{Chapter 7}\label{ch:chapter7}}

This chapter contains a summary of the context in which your research is set.

Imagine you are writing for your fellow PhD students. Topics that are well-known to them do not have to be included here. But things that they may not know about should be included.

Resist the temptation to discuss everything you've read in the last few years. And you are not writing a textbook either. This chapter is meant to provide the background necessary to understand the material in subsequent chapters. Stick to that.

You will need to organize the literature review around themes, and within each theme provide a story explaining the development of ideas to date. In each theme, you should get to the point where your ideas will fit in. But leave your ideas to later chapters. This way it is clear what has been done beforehand, and what new contributions you are making to the research field.

All citations should be done using markdown notation as shown below. This way, your bibliography will be compiled automatically and correctly.

\hypertarget{ch:summary}{%
\chapter{Summary}\label{ch:summary}}

This chapter contains a summary of the context in which your research is set.

\hypertarget{acknowledgements}{%
\chapter*{Acknowledgements}\label{acknowledgements}}
\addcontentsline{toc}{chapter}{Acknowledgements}

\newpage

\textbf{Acknowledgements}

\hypertarget{overview-of-completed-training-activities}{%
\chapter*{Overview of completed training activities}\label{overview-of-completed-training-activities}}
\addcontentsline{toc}{chapter}{Overview of completed training activities}

\textbf{Discipline specific activities}

\hypertarget{about-the-author}{%
\chapter*{About the author}\label{about-the-author}}
\addcontentsline{toc}{chapter}{About the author}

\hypertarget{list-of-publications}{%
\chapter*{List of publications}\label{list-of-publications}}
\addcontentsline{toc}{chapter}{List of publications}

\hypertarget{last-page}{%
\chapter*{last page}\label{last-page}}
\addcontentsline{toc}{chapter}{last page}

\newpage

\appendix

\hypertarget{additional-stuff}{%
\chapter{Additional stuff}\label{additional-stuff}}

You might put some computer output here, or maybe additional tables.

Note that line 5 must appear before your first appendix. But other appendices can just start like any other chapter.

\printbibliography[heading=bibintoc]



\end{document}
